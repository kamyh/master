\chapter{Parallelisation python3}
\label{ch:parallel}

Nous allons principalement nous servir de deux fonctionnalités afin de rendre une partie, critique, du code parallèle. En effet, la recherche de domaine protéinique, à l'aide de \emph{HMMER Scan} est une opération très longue.

Nous allons rendre parallèle les appels à un conteneur Docker encapsulant une version locale de \emph{HMMER} et la lecture des résultats.

Pour plus d'informations sur la transformation du code, que ce soit l'ajout de la philosophie objet ou au niveau des optimisations, veuillez consulter le \cf{ch:app}

\section{Code de base}

Dans ce sous-chapitre il est question de comment paralléliser un code en python 3, lorsque l'on a un traitement à appliquer sur un ensemble de données de même type.

Le code suivant montre comment utiliser la librairie \emph{multiprocessing} et les \emph{pool}.

\lstset{language=python}
\begin{lstlisting}[frame=single]
from multiprocessing import Pool

def f(x):
    return x*x

if __name__ == '__main__':
    p = Pool(processes=5)
    print(p.map(f, [1, 2, 3]))
\end{lstlisting} 

Ce qui se passe dans ce code c'est que l'on \emph{map} une fonction sur un tableau de données. Ce code va donc appliquer la fonction carrée sur chaque élément du tableau et renvoyer les résultats dans un tableau.

Il est également possible de passer à la fonction \emph{Pool.map()} des paramètres plus complexes:

\lstset{language=python}
\begin{lstlisting}[frame=single]
from multiprocessing import Pool

def f(x, y):
    return x+y

if __name__ == '__main__':
    p = Pool(processes=5)
    print(p.starmap(f, [[1,2],[3,4],[5,6]]))
\end{lstlisting} 

La ligne suivante, permet de définir le nombre de coeurs \gls{cpu} que l'on souhaite utiliser: 

\begin{lstlisting}[frame=single]
p = Pool(processes=5)
\end{lstlisting} 

Attention la fonction \emph{starmap} n'est disponible que depuis python 3.3.

Retrouver ces codes dans le dossier \emph{$sources/exemple_3$}. Dans ce chapitre nous n'aborderons que les éléments nécessaires à la compréhension et à l'utilisation du code produit dans cette thèse. Pour plus d'informations sur la librairie multiprocessing cf. https://docs.python.org/2/library/multiprocessing.html.

Dans un des scripts du code de la thèse \thLeite , des séquences protéiniques sont traitées en faisant appel à l'\gls{api} de HMMER. Dans le \cf{ch:app} nous verrons comment cet appel a été remplacé par un conteneur Docker.






































