\chapter{Déploiement et execution}
\label{ch:setup}

Dans ce chapitre, nous allons aborder la manière dont l'environnement et l'application doivent être déployés.

\section{Obtention des sources}

Le projet réalisé durant cette thèse a été développé en utilisant les gestionnaires de code source \emph{git} par l'intermédiaire de la plateforme \emph{Github.com}. Le projet est en accès public à l'adresse, \url{https://github.com/kamyh/master.git}.

\begin{figure}[H] 
\centering 
\begin{lstlisting}[frame=single]
$ git clone https://github.com/kamyh/master.git
$ cd master/developpement/dockers/core/data-hmm/
$ sh get_pfam_hmm.sh
$ cd ../../database/data/
$ wget https://www.dropbox.com/s/mzt9pxpfnvxl3wa/bacteriaVD.sql?dl=0
$ mv bacteriaVD.sql?dl=0 bacteriaVD.sql
\end{lstlisting} 
\caption[Code - Obtention de sources]{Obtention des sources}
\label{fig:getSources} 
\end{figure}

Le code \ref{fig:getSources} permet d'obtenir les sources, mais également la base de données \emph{Pfam} et la base de données des bactéries.

\section{Edition de la configuration}

Il faut tout d'abord créer au moins un fichier de configuration en copiant le fichier de configuration par défaut.

\begin{figure}[H] 
\centering 
\begin{lstlisting}[frame=single]
$ cd ../../../inphinity/v_0.5
$ cp configs/config.ini.example config.ini
$ sudo nano configs/config.ini
\end{lstlisting} 
\caption[Code - Edition de la configuration]{Edition de la configuration}
\label{fig:renameConfig} 
\end{figure}

Il est impératif de modifier la valeur de \emph{$path\_to\_core$} afin qu’elle soit correcte par rapport à la machine hôte.

\section{Execution de la composition Docker}

Il est à présent possible d'executer l'environnement de notre application.

\begin{figure}[H] 
\centering 
\begin{lstlisting}[frame=single]
$ cd ../../compose/
$ sudo sh run.sh
$ docker ps
\end{lstlisting} 
\caption[Code - Execution composition]{Execution de la composition Docker}
\label{fig:execCompose} 
\end{figure}

Une fois les commandes \ref{fig:execCompose} executées, \emph{docker ps} devrait retourner ceci:

\begin{figure}[H] 
\centering 
\begin{lstlisting}[frame=single]
CONTAINER ID        IMAGE               COMMAND                  CREATED             STATUS              PORTS                              NAMES
6817333ab720        compose_core        "/bin/bash"              11 seconds ago      Up 7 seconds                                           inphinity-core
b494735f1440        compose_database    "tini -- /bin/bash /o"   12 seconds ago      Up 10 seconds       3309/tcp, 0.0.0.0:3309->3306/tcp   inphinity-database
\end{lstlisting} 
\caption[Code - Execution composition vérification]{Execution de la composition Docker vérification}
\label{fig:execComposeCheck} 
\end{figure}

L'application n'est pas executée automatiquement pour le moment, mais en cas de déploiement en prodution, cela est facile à modifier.

Pour le moment, il faut se connecter au conteneur du contrôleur et executer l'application:

\begin{figure}[H] 
\centering 
\begin{lstlisting}[frame=single]
$ docker exec -it inphinity-core /bin/bash
$ python3 inphinity/v_0.5/app.py 
\end{lstlisting} 
\caption[Code - Execution Application]{Execution de l'application}
\label{fig:execComposeCheck} 
\end{figure}



















