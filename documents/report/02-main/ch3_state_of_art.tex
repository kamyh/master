\chapter{Etats de l'art}
\label{ch:state_art}

\section{introduction}
Dans ce chapitre nous aborderons les différentes pistes envisagées afin de remplir les objectifs fixés dans cette thèse, comme listé dans l'introduction (\cf{ch:introduction}).

Avant toutes choses il à fallu se mettre au niveau et comprendre la thèse [TODO cf name diogo thesis].

\section{Automatisation}
En terme d'automatisation, une pratique bien courante chez les developpeur s'agit à utiliser des script bash afin de pouvoir executer une certain nombre de commande et de code succecivement. Bien que cette méthode présente l'avantage d'etres simple, il suffit d'une console UNIX et d'un editeur de texte, elle présente un défault majeur. En effet, le developpeur du script contrôle quel commande et code sont executé et peut également definir des paramètres pour ceux-ci, mais il ne peu pas contrôlé l'environnement d'execution.

Une façcon de faire, en pleine essort depuis quelque temps, est l'utilisation de la platforme Docker. Il s'agit d'un logiciel de containerisation. C'est-à-dire la création de brique d'application qui mise en communes permette de réaliser un application global. De plus, le développement d'une telle solution, permet un partage facilité grâce à un déploiement facilité et autonome. Pour d'avantages d'explication sur le sujet je vous renvoi au chapitre  \cf{ch:docker}.

Vous l'aurez bien compris, le choix qui à été fait est celui de l'utilisation de Docker.

\section{Configuration}

En ce qui concerne la recherche d'une méthode afin de réalisé facilement des fichiers de configurations pré-créées, beaucoup de solutions existent. Ces différentes méthodes sont plus ou moins flexible aux modifications.

Les fichiers de configurations dont il est question ici, sont spécifique à la partie python du code qui sera executé par notre application \cf{ch:app}
 
%configparser

\section{Parallélisation}
%Docker parallelisation
%Docker Swarm
%Biopython parallelisation
%Spark
	%%local
	%%Spark on Amazone cloud
	

\subsection{Simple}
\subsection{Avancée}

\section{Optimisations}
%python3
%Cython

\section{Hmmer}
%Hmmer

\section{conclusion}
%Docker + python
%premier tests divers