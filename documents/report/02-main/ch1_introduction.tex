\chapter{Introduction}
\label{ch:introduction}

Ce travail à été réalisé la la suite d'une autre thèse de master dont l'interet était de prouvé la pertinence d'une méthode d'analyse par \emph{machine learning}. En effet, il s'agit d'une méthode permettant [TODO: completer].

Dans la présente thèse il est question de mettre en place plusieurs aspect permettant l'anrichissement du processus d'analyse de la thèse \emph{[TODO: these name of diogo]}. 

Notamment, l'utilisation de python 3 pour remplacer l'utilisation de python2, moins efficace. %TODO: proof

De plus, ont souhait être capable d'automatiser le lancement de "l'application" et par la même occasion rendre le déploiement facile et unifier quelque soit la machine cible, pour autant qu'elle utilise le système d'exploitation Linux.

Troisièmement, on souhaite pouvoir lancer l'analyse pour différentes configuration, créées à l'avance.

Afin de réaliser ces objectif une première phase du travail à consisté à realiser des états de l'art pour les différents domaines utilisé (cf. Chapitre 3). %TODO: link to chapter 3


%%%%%%%%%%%%%%%%%%%%%%%%
%TODO: Information concernant la volonté de réaliser un Docker for Bio-Informatique
%%%%%%%%%%%%%%%%%%%%%%%%