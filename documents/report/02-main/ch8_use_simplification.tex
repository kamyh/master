\chapter{Simplification d'usage}
\label{ch:simple}

Le projet étant relativement complexe, il est important de connaitre certaines astuces permettant de simplifier son utilisation.

\section{Commandes et alias}

La commande de la figure \ref{fig:cmdDockerExec}, permet de se connecter à un conteneur, par exemple, celui des bases de données.

\begin{figure}[H] 
\centering 
\begin{lstlisting}[frame=single]
$ docker exec -it <<CONTAINER NAME/ID>> /bin/bash
$ docker exec -it inphinity-core /bin/bash
$ docker exec -it inphinity-database /bin/bash
\end{lstlisting} 
\caption[Code - Commande docker exec]{Commande docker exec}
\label{fig:cmdDockerExec} 
\end{figure}

Le première alias de la figure \ref{fig:aliases}, permet d'afficher des informations sur l'état du conteneur de base de données.

\begin{figure}[H] 
\centering 
\begin{lstlisting}[frame=single]
alias inphinitydbstate="docker exec -it inphinity-database mysql -h inphinity-database -u admin -proot -e \"use phage_bact; SELECT (SELECT count(ProtDomId) as doms_found from phage_bact.PROTDOM) as doms_found, (SELECT count(id) as doms_done from phage_bact.progress) as doms_done, (SELECT count(Score_Inter_Id) from phage_bact.Score_interactions) as interactions_found;\""

alias inphinitylogs="docker exec -it inphinity-core tail -100 /tmp/logs_inphinity.txt; echo ''"
alias inphinitylogsnormal="docker exec -it inphinity-core tail -1000 /tmp/logs_inphinity.txt |grep NORMAL; echo ''"
\end{lstlisting} 
\caption[Code - Alias]{Alias}
\label{fig:aliases} 
\end{figure}

Les deux autres alias de la figure \ref{fig:aliases}, permettent d'afficher certaines informations des \emph{logs}, directement depuis la machine hôte.

\newpage
\section{Scripts}

Le dossier \emph{$developpement/scripts/$}, contient des scripts utiles à l'utilisation de ce projet.

\begin{itemize}
\item \emph{$install\_docker.sh$}, permet d'installer \emph{Docker} et \emph{Docker Compose} automatiquement sur n'importe quelle machine Linux.
\item \emph{deploy.sh}, permet de déployer l'environnement de l'application, de récupérer les codes source, de lancer l'environnement et de se connecter au conteneur Docker du contrôleur.
\item \emph{$docker\_ps\_lookup.sh$}, est simplement un script infini, permettant de surveiller l'état de conteneurs \emph{Docker} en cours d'exécution.
\item \emph{$infos\_lookup.sh$}, est simplement un script infini, permettant de surveiller l'état de l'exécution de l'application.
\end{itemize}




























