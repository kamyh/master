\chapter{Conclusion}
\label{ch:conclusion}

Ce projet propose une solution qui prend le partit d'utiliser la platforme \emph{Docker}, qui offre beaucoup d'avantages. Nottament en terme de déploiement sur différente machine hôte (différent developpeur/utilisateurs). De plus, même si pour le moment l'application n'est que dans une optique de démonstration, le fait de développer en utilisant \emph{Docker} permet de passer tres rapidement en phase de production. Il faut également noté que \emph{Docker} est souvent utiliser pour faire du déploiment continue.

Le code, précédement (thèse \thLeite) développé sous forme de scripts qui necessitait de nombreuses entrées utilisateurs, est maintenant complétement automatique. De plus, l'application peut etres executé plusieurs fois, en fournissant autant de fichiers de configuration que l'on souhaite. Chacune de ces executions est succeptible de produire un \emph{dataset} spécifique. Mais une execution peut aussi executer que certaines phase du traitement (\thLeite).

Travailler sur un code déjà existant, est une expérience très interessante, même si cela présente également des désavantages. Un avantage important, est le fait d'avoir la liberté de se concentré sur certains aspect uniquement et de cette manière de les approfondire davantage. C'est ce qui à été fait durant se travail avec l'idée de parallélisation et l'utilisation de la platforme \emph{Docker}.
