\chapter{Conclusion}
\label{ch:conclusion}

Ce projet propose une solution qui prend le parti d'utiliser la plateforme \emph{Docker}, qui offre beaucoup d'avantages. Notamment en termes de déploiement sur différentes machines hôtes (différents développeurs/utilisateurs). De plus, même si pour le moment l'application n'est que dans une optique de démonstration, le fait de développer en utilisant \emph{Docker} permet de passer très rapidement en phase de production. Il faut également noter que \emph{Docker} est souvent utilisé pour faire du déploiement continu.

Le code, précédemment (thèse \thLeite) développé sous forme de scripts qui nécessitaient de nombreuses entrées utilisateurs, est maintenant complètement automatique. De plus, l'application peut être exécutée plusieurs fois, en fournissant autant de fichiers de configuration que l'on souhaite. Chacune de ces exécutions est susceptible de produire un \emph{dataset} spécifique. Mais on peut aussi n'exécuter que certaines phases du traitement (\thLeite).

Travailler sur un code déjà existant est une expérience très intéressante, même si cela présente également des désavantages. Un avantage important est le fait d'avoir la liberté de se concentrer sur certains aspects uniquement et de cette manière de les approfondir davantage. C'est ce qui a été fait durant ce travail avec l'idée de parallélisation et l'utilisation de la plateforme \emph{Docker}.
