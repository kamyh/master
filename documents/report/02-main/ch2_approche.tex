\chapter{Approche}
\label{ch:approche}

Dans la vie professionnel de tous developpeur ou ingenieur il est courant de reprendre un projet déjà existant, et ainsi de ne pas commencer votre propre travail à partir de rien. Comme précisé dans l'introduction, se travail se place dans la continuité d'une autre thèse de master.

Dans ce chapitre nous allons rapidement traiter la méthodologie mise en place pour aborder la réalisation de cette thèse. En effet, plusieurs phases distinctes de travail ont été necessaire durant ce travail.

Premièrement, il à fallu reprendre la thèse [TODO: ref these diogo] et comprendre se qu'il y à été fait. Les information concernant la thèse de Mr.Leite Diogo necessaire à la compréhension de se travail ont été abordé dans l'introduction, pour d'avantage d'information veuillez consulter la thèse en question.

Deuxièmement, une fois les objectifs de thèse fixé il à été important de réaliser un état de l'art des différentes technologies et aspects techniques succeptible d'êtres utilisées dans la présente thèse, voir chapitre~\ref{ch:state_art}-\nameref{ch:state_art} .

Troisièmement, c'est uniquement après ces deux premières phases de travail que le développement à pu commencer, voir chapitre~\ref{ch:parallel}-\nameref{ch:parallel} et chapitre~\ref{ch:app}-\nameref{ch:app}.

Finalement, le temps de travail étant limité, il faut penser aux utilisation futurs de ce qui à été developpé. Ceci passe notamment par l'utilisation de l'application réalisée dans ce travail de manière simple voir chapitre~\ref{ch:simple}-\nameref{ch:simple}, mais aussi par les améliorations possible à cette thèse, voir chapitre~\ref{ch:improve}-\nameref{ch:improve}.

Il faut aussi préciser que certains résultats et métrique ont été réalisé et sont regroupé dans le chapitre~\ref{ch:results}-\nameref{ch:results}.

%%%%%%%%%%%%%%%%%%%%%%%%%%%%%%%%%%%%%%%%%%%%%%%%%%%%%%
%TODO: reférence sur les chapitres