%\begingroup
%\let\cleardoublepage\clearpage


% English abstract
\cleardoublepage
\chapter*{Abstract}
%\markboth{Abstract}{Abstract}
\addcontentsline{toc}{chapter}{Abstract (English/Français)} % adds an entry to the table of contents
% put your text here

Nowadays, drugs are losing ground to bacteria. In deed, bacteria become resistant. A possible alternative to the use of antibiotics is phagotherapy. It's about using viruses that eat bacteria to heal infection. One of the advantages of phagotherapy is that these viruses are very specific and only target specific bacteria.

This project is following another master thesis, \thLeite . This first thesis is about trying to establish a method, to quickly identify bacteriophages capable to eat a specific targeted bacteria. For now this procedure is done manually in laboratories by testing all possible combinations of interaction viruses and bacteria.

In the thesis \thLeite , a solution has been proposed. The present thesis aims to improve this solution. Particularly in terms of performance and its use.

To do this, this thesis explores a solution that sets up a Docker environment. Docker is a containerization software. Docker can automate the deployment of complex applications.

The application produced during this work allows to launch a Docker architecture and to run the application autonomously without the need for user interaction. In addition, it is possible to provide as many configuration files as you want, thus producing as many datasets as configuration files.

\vskip0.5cm
Key words: Docker, Environment, Configuration
%put your text here


% French abstract
\begin{otherlanguage}{french}
\cleardoublepage
\chapter*{Résumé}

Aujourd'hui les médicaments perdent du terrain sur les bactéries. En effet, les bactéries deviennent résistantes. Une des alternatives possibles à l'utilisation des antibiotiques est la phagothérapie. Il s'agit d'utiliser des virus \emph{mangeurs} de bactéries pour soigner les infections bactériennes. Un des avantages majeurs à la phagothérapie est que les virus sont très spécifiques et ne s'attaquent qu'aux bactéries ciblées. 

Ce projet fait suite à une autre thèse de master, \thLeite . Dans cette première thèse, il est question de tenter d'établir une méthode, afin de rapidement identifier les bactériophages capables de s'attaquer à une bactérie cible. Pour l'instant cette procédure est effectuée manuellement en laboratoire en testant les combinaisons possibles d'interaction virus/bactéries.

Dans la thèse \thLeite , une solution été proposée. Cette thèse a pour objectifs d'améliorer cette solution. Notamment au niveau des performances et de son utilisation.

Pour se faire, cette thèse explore une solution qui met en place un environnement Docker. Docker est un logiciel de conteneurisation. Docker permet s'automatiser le déploiement d'applications complexes.

L'application produite durant ce travail permet de lancer une architecture Docker et d'exécuter l'application de manière autonome sans nécessité d'interaction utilisateur. De plus, il est possible de fournir autant de fichiers de configurations que l'on souhaite, et de produire ainsi autant de \emph{dataset} que de fichiers de configurations.

Mots Clés: Docker, Environnement, Configuration
\end{otherlanguage}


%\endgroup			
%\vfill
