%\begingroup
%\let\cleardoublepage\clearpage


% English abstract
\cleardoublepage
\chapter*{Abstract}
%\markboth{Abstract}{Abstract}
\addcontentsline{toc}{chapter}{Abstract (English/Français)} % adds an entry to the table of contents
% put your text here
\lipsum[1-2]
\vskip0.5cm
Key words: 
%put your text here


% French abstract
\begin{otherlanguage}{french}
\cleardoublepage
\chapter*{Résumé}

Aujourd'hui les médicaments perdent du terrain sur les bactéries. En effet, les bactéries deviennent résistantes. Une des alternatives possibles à l'utilisation des antibiotiques est la phagothérapie. Il s'agit d'utiliser des virus \emph{mangeurs} de bactéries pour signer les infections bactériennes. Un des avantages majeurs à la phagothérapie est que les virus sont très spécifiques et ne s'attaque qu'aux bactéries ciblées. 

Ce projet fait suite à une autre thèse de master, \thLeite . Dans cette première thèse, il est question de tenter d'établir une méthode, afin de rapidement identifier les bactériophages capables de s'attaquer à une bactérie cible. Pour l'instant cette procédure est effectuée manuellement en laboratoire en testant les combinaisons possibles d'interaction virus/bactéries.

Dans la thèse \thLeite , une solution été proposée. Cette thèse a pour objectifs d'améliorer cette solution. Notamment au niveau des performances et de son utilisation.

Pour se faire cette thèse explore une solution qui met en place un environnement Docker. Docker est un logiciel de conteneurisation. Docker permet s'automatiser le déploiement d'application complexe de manière automatisée.

L'application produite durant ce travail permet de lancer une architecture Docker et d'exécuter l'application de manière autonome sans nécessité d'interaction utilisateur. De plus, il est possible de fournir autant de fichiers de configurations que l'on souhaite, et de produire ainsi autant de \emph{dataset} que de configurations.

\end{otherlanguage}


%\endgroup			
%\vfill
